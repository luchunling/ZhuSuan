\documentclass{article}

\usepackage{graphicx}
\usepackage{xeCJK}
\usepackage{bm}
\usepackage{amsmath,amsthm,amssymb,amsfonts}
\usepackage{color}
\usepackage{geometry}
\geometry{left=3.2cm,right=3.2cm,top=3.2cm,bottom=3.2cm}

%给文字加颜色,用法:
%{\color{red}{I love you}}

%\usepackage{ntheorem}
\newtheorem{them}{定理}[subsection]
\newtheorem{defn}{定义}[subsection]
\newtheorem{lemm}{引理}[subsection]

%\setCJKmainfont[BoldFont = 黑体]{宋体}
%\setlength{\parindent}{2em}
%如果不要缩进 用\noindent
\title{Problem Assignment \#3}

\author{Statistical Machine Learning}

\date{}

\begin{document}

\maketitle

\section{Problem 1. 2-D Gaussian Mixture}

\subsection{Model}

生成模型如下

\begin{eqnarray}
	Z_i &\sim& \mathrm{Discrete}(\pi) \\
	X_i \lvert Z_i &\sim& \mathcal{N}(\mu_{Z_i}, \sigma^2_{Z_i})
\end{eqnarray}

其中$Z_i$和$X_i$是随机变量,$\{x_i\}_{i=1}^N$是$X_i$的具体的观测。$\mathrm{Discrete}(x)$为取值为$\{1,2,\cdots,K\}$的随机变量,并且满足$\mathbb{P}(Z_i)=\pi_i$。$\pi \in \mathbb{R}^K, \mu_i \in \mathbb{R}^D, \sigma^2_i \in \mathbb{R}^D$是需要学习的参数,其中$i \in \{1, 2, \cdots, K\}$。$N$、$K$和$D$是固定的常数,都是正整数。

在此问题中我们固定$D=2$,需要使用variational inference的方法求出参数$\pi \in \mathbb{R}^K, \mu_i \in \mathbb{R}^D, \sigma^2_i \in \mathbb{R}^D$的极大似然估计。

\subsection{Requirements}

代码需要包括:

\begin{itemize}
	\item [1.] 在ZhuSuan中实现生成模型(model),并使用你写的模型生成数据(这一部分可以参考vae的教程的生成图像的部分)。
	\item [2.] 在ZhuSuan中设计合理的variational posterior distribution(q\_net)。
	\item [3.] 基于ZhuSuan的框架实现整个算法(model-inference-learning),用[1]步生成的数据进行训练(推荐使用zs.rws)。
	\item [4.] 对于结果的可视化(可以使用matplotlib)。
\end{itemize}

报告需要包括:

\begin{itemize}
	\item [1.] 对于整个问题的形式化描述(按照model-infernece-learning的框架),尤其是设计的variational posterior distribution的描述与分析。
	\item [2.] 固定$N=100$, $K=3$,生成数据并完成训练,从数值角度和可视化的角度展示你的结果(需要有具体的图)。
	\item [*3.] 探讨$N, K$以及参数的真值(比如$\mu_i$之间是否比较近)的变化对你的方法产生结果的影响。
	\item [*4.] 如果设计了多个variational posterior,比较不同的variational posterior的结果。
\end{itemize}

关于数据生成:先固定$N$,$D$,$K$,随机生成或者手动选取参数的真值$\pi \in \mathbb{R}^K, \mu_i \in \mathbb{R}^D, \sigma^2_i \in \mathbb{R}^D$,使用生成模型生成$N$组数据$\{x_i\}_{i=1}^N$,在求解极大似然估计的时候应该只用生成的数据$\{x_i\}_{i=1}^N$而“忘记”参数的真值,最后比较极大似然估计的估计值和参数的真值。

\section{Problem 2. Gaussian Mixture VAE}

我们考虑将Gaussian Mixture和VAE结合起来,生成模型如下

\begin{eqnarray}
	Z_i &\sim& \mathrm{Discrete}(\pi) \\
	H_i \lvert Z_i &\sim& \mathcal{N}(\mu_{Z_i}, \sigma^2_{Z_i}) \\
	X_i \lvert H_i &\sim& \mathrm{Bernoulli}(\sigma_{NN}(H_i))
\end{eqnarray}

其中$Z_i$, $H_i$和$X_i$是随机变量,$\{x_i\}_{i=1}^N$是$X_i$的具体的观测。$\mathrm{Discrete}(x)$为取值为$\{1,2,\cdots,K\}$的随机变量,并且满足$\mathbb{P}(Z_i)=\pi_i$。$Z_i \in \mathbb{R}^K$, $H_i \in \mathbb{R}^D$, $X_i \in \{0,1\}^{784}$。$\sigma_{NN}(H_i)$为神经网络,构成一个从$\mathbb{R}^D$到$[0,1]^{784}$的映射。以 $\pi \in \mathbb{R}^K, \mu_i \in \mathbb{R}^D, \sigma^2_i \in \mathbb{R}^D$和神经网络$\sigma_{NN}(\cdot)$中的参数是需要学习的参数,其中$i \in \{1, 2, \cdots, K\}$。$N$、$K$和$D$是固定的常数,都是正整数。

在此问题中我们固定$D=40$,$K=10$,使用MNIST数据集(具体使用可以看examples.utils中的dataset.py和examples中的vae.py)需要使用variational inference的方法求出参数$\pi \in \mathbb{R}^K, \mu_i \in \mathbb{R}^D, \sigma^2_i \in \mathbb{R}^D$和神经网络$\sigma_{NN}(\cdot)$中的参数的极大似然估计,建议使用Reweighted Wake Sleep算法。

\subsection{Requirements}

代码需要包括:

\begin{itemize}
	\item [1.] 在ZhuSuan中实现生成模型(model)。
	\item [2.] 在ZhuSuan中设计合理的variational posterior distribution(q\_net)。
	\item [3.] 基于ZhuSuan的框架实现整个算法(model-inference-learning),用MNIST数据集进行训练(推荐使用zs.rws)。
	\item [4.] 对于结果的可视化(画出samples,固定取$Z_i=\{1,2,\cdots,K\}$不变,对于每一种用生成模型生成多个$X_i$,观察其聚类效果)。
\end{itemize}

报告需要包括:

\begin{itemize}
	\item [1.] 对于整个问题的形式化描述(按照model-infernece-learning的框架),尤其是设计的variational posterior distribution的描述与分析。
	\item [2.] 对于数值结果的分析和samples的可视化展示与分析。
	\item [*3.] 如果设计了多个variational posterior,比较不同的variational posterior的结果。
\end{itemize}

\end{document}